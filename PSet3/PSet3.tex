\documentclass{scrreport}
\usepackage[utf8]{inputenc}
\usepackage{mathtools, amssymb}
\DeclarePairedDelimiter\abs{\lvert}{\rvert}%
\usepackage{parskip}
\usepackage{hyperref}

\title{Complex Analysis Problem Set 3}
\author{D. Zack Garza}
\date{\today}

\begin{document}

\maketitle
\tableofcontents

\chapter{Tie}

\section{1}
Prove that if $f$ has two Laurent series expansions,
\begin{align}
    \sum c_n(z-a)^n \quad \sum c_n'(z-a)^n
\end{align}
then $c_n = c_n'$.

\section{2}
Find Laurent series expansions of
\begin{align}
    \frac{1}{1-z^2} + \frac{1}{3-z}
\end{align}

How many such expansions are there?
In what domains are each valid?

\section{3}

Let $P, Q$ be polynomials with no common zeros.
Assume $a$ is a root of $Q$. 
Find the principal part of $P/Q$ at $z=a$ in terms of $P$ and $Q$ if $a$ is (1) a simple root, and (2) a double root.

\section{4}
Let $f$ be non-constant, analytic in $\abs{z} > 0$, where $f(z_n) = 0$ for infinitely many point $z_n$ with $\lim z_n \to 0$.

Show that $z=0$ is an essential singularity for $f$.

\begin{quote}
    Example: $f(z) = \sin(1/z)$.
\end{quote}

\section{5}
Show that if $f$ is entire and $\lim_{z\to\infty}f(z) = \infty$, then $f$ is a polynomial.

\section{6}

\subsection{a}
Show that
\begin{align}
    \int_0^{2\pi} \log\abs{1 - e^{i\theta}}~d\theta = 0
\end{align}

\subsection{b}
Show that this identity is equivalent to SS 3.8.9.

\section{7}
Let $0<a<4$ and evaluate
\begin{align}
    \int_0^\infty \frac{x^{\alpha-1}}{1+x^3} ~dx
\end{align}

\section{8}
Prove the fundamental theorem of Algebra using

\subsection{a}
Rouche's Theorem.

\subsection{b}
The maximum modulus principle.

\section{9}
Let $f$ be analytic in a region $D$ and $\gamma$ a rectifiable curve in $D$ with interior in $D$.

Prove that if $f(z)$ is real for all $z\in \gamma$, the $f$ is constant.

\section{10}
For $a> 0 $, evaluate
\begin{align}
    \int_0^{\pi/2} \frac{d\theta}{a + \sin^2 \theta}
\end{align}

\section{11}
Find the number of roots of $p(z) = 4z^4 - 6z + 3$ in $\abs{z} < 1$ and $1 < \abs{z} < 2$ respectively.

\section{12}
Prove that $z^4 + 2z^3 -2z + 10$ has exactly one root in each open quadrant.

\section{13}
Prove that for $a> 0$, $z\tan z - a$ has only real roots.

\section{14}
Let $f$ be nonzero, analytic on a bounded region $\Omega$ and continuous on its closure $\overline \Omega$.

Show that if $\abs{f(z)} \equiv M$ is constant for $z\in \partial \Omega$, then $f(z) \equiv Me^{i\theta}$ for some real constant $\theta$.

\chapter{Stein and Shakarchi}
\section{S&S 3.8.1}

Using Euler's formula
\begin{equation}
    \sin(\pi z) = \frac 1 {2i}( e^{i\pi z} - e^{-i\pi z} )
\end{equation}
show that the complex zeros of $\sin(\pi z)$ are exactly the integers, each of order one.
Calculate the residue of $\frac 1 {\sin(\pi z)}$ at $z=n\in \mathbb{Z}$.

\section{S&S 3.8.2}
Evaluate the integral
\begin{equation}
    \int_{\mathbb{R}} \frac{dx}{1+x^4} 
\end{equation}
What are the poles of the integrand?

\section{S&S 3.8.4}
Show that
\begin{align}
    \int_{\mathbb R} \frac{x\sin x}{x^2 + a^2} = \frac{\pi e^{-a}}{a} \quad a > 0
\end{align}

\section{S&S 3.8.5}
Show that for $\xi \in \mathbb{R}$,

\begin{align}
    \int_{\mathbb R} \frac{e^{2\pi i x \xi}}{(1+x^2)^2} = \frac \pi 2 ( 1 + 2\pi |\xi|)e^{-2\pi |\xi|}
\end{align}

\section{S&S 3.8.6}
Show that
\begin{align}
    \int_{\mathbb R} \frac{dx}{(1+x^2)^{n+1}} = \frac{1\cdot 3\cdot \cdots (2n-1)\pi }{2\cdot 4\cdots (2n}
\end{align}

\section{S&S 3.8.7}
Show that for $a > 1$,
\begin{align}
    \int_0^{2\pi} \frac{d\theta}{(a+\cos \theta)^2} = 
    \frac{2\pi a}{(a^2-1)^{3/2}}
\end{align}

\section{S&S 3.8.8}
Show that if $a, b\in \mathbb{R}$ with $a > \abs{b}$ then
\begin{align}
    \int_0^{2\pi} \frac{d\theta}{a + b\cos \theta} =
    \frac{2\pi a}{\sqrt{a^2-b^2}}
\end{align}

\section{S&S 3.8.9}
Show that 
\begin{align}
    \int_0^1 \log(\sin \pi x) ~dx = -\log 2
\end{align}

\section{S&S 3.8.10}
Show that if $a> 0$
\begin{align}
\int_0^{\infty} \frac{\log x}{x^2 + a^2} ~dx =\frac{\pi \log a}{2a}
\end{align}

\section{S&S 3.8.14}
Prove that if $f$ is entire and injective, then $f(z) = az + b$ with $a,b\in \mathbb{C}$ with $a\neq 0$.

\begin{quote}
  Hint: apply the Casorati-Weierstrass theorem to $f(1/z)$.  
\end{quote}


\section{S&S 3.8.15}
Use the Cauchy inequalities or the maximum modulus principle to solve the following problems:

\subsection{a}
If $f$ is entire and for all $R> 0$, there are constants $A, B > 0$ such that $\sup_{\abs{z} = R} \abs{f(z)} \leq AR^k + B$, then $f$ is a polynomial of degree less than $k$.

\subsection{b}
Show that if $f$ is holomorphic on the unit disk, is bounded, and converges to zero uniformly in the sector $\theta \leq \arg z \leq \phi$ as $\abs{z} \to 1$, then $f \equiv 0$.

\subsection{c}
Let $w_1, \cdots, w_n$ be points on $S^1 \subset\mathbb{C}$. Show that there exists a point $z \in S^1$ such that 
$$
\prod_{i=1}^n \abs{z-w_i} \geq 1
$$.

Conclude that there exists a point $w\in S^1$ such that 
$$
\prod_{i=1}^n \abs{w-w_i} = 1.
$$ 

\subsection{d}
Show that if $f$ is entire and $\Re(f)$ is bounded, then $f$ is constant.

\section{S&S 3.8.17}
Let $f$ be non-constant, and holomorphic in an open set containing the open unit disc.

\subsection{a}
Show that $\abs{z} = 1 \implies \abs{f(z)} = 1$, then the image of $f$ contains the unit disc.

\begin{quote}
    Hint: Show that $f(z) = w_0$ has a root for every $w_0 \in \mathbb{D}$, for which it suffices to show that $f(z) = 0$ has a root, then use the maximum modulus principle.
\end{quote}

\subsection{b}
Show that if $\abs{z} \geq 1 \implies \abs{f(z)} = 1$ \textbf{and} there exists a point $z_0 \in \mathbb{D}$ such that $\abs{f(z_0)} < 1$, then the image of $f$ contains the unit disc.

\section{S&S 3.8.19}

Prove the maximum modulus principle for harmonic functions; i.e.,

\subsection{a}
If $u$ is a non-constant real-valued harmonic function on $\Omega$, then $u$ can not attain its extrema on $\Omega$.

\subsection{b}
Suppose $\Omega$ has compact closure $\overline \Omega$, then if $u$ is harmonic on $\Omega$ and continuous on $\overline \Omega$, then
\begin{align}
    \sup_{z\in \Omega} \abs{u(z)} \leq \sup_{z\in \overline \Omega - \Omega} \abs{u(z)}
\end{align}

\begin{quote}
    Hint: to prove (a), assume that $u$ attains a local maximum at $z_0$, and let $f$ be holomorphic near $z_0$ with $u = \Re(f)$, then show that $f$ is not open.
    Part (b) is a direct consequence.
\end{quote}

\end{document}
